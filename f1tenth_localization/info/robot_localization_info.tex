\documentclass[a4paper]{article}
\usepackage{graphicx} 
\usepackage{amsmath} 
\usepackage{amsfonts}
\usepackage{cleveref}
\addtolength{\topmargin}{-20mm}
\addtolength{\textheight}{20mm}
\usepackage[T1]{fontenc}
\usepackage[utf8]{inputenc}
\usepackage[]{algorithm2e}
\usepackage[obeyspaces]{url}

\newcommand \eq[1]{\begin{equation}\begin{aligned}#1 \end{aligned}\end{equation}}

\begin{document}
\title{Robot localization description}
\author {Linus Härenstam-Nielsen}
\maketitle


\subsection*{EKF-node}
The \texttt{ekf\_localization\_node} implements the following algorithm:
\begin{enumerate}
	\item wait for new measurements
	\item \textbf{predict} next state and error covariance using a second order omnidirectional model 
	\item \textbf{correct} the prediction using measurements and update the error covariance
	\item go to step 1.
\end{enumerate}

\subsubsection*{Prediction step}
When using \path{ekf_localization_node} in 2D-mode the next state $\tilde{s}_{t+1}$ is predicted from the current state estimate $s_t$ using a second order omnidirectional model:

\eq{
\label{eqn:state_transition}
\begin{bmatrix}
x\\y\\\dot{x}\\\dot{y}\\\ddot{x}\\\ddot{y}\\\varphi\\\dot{\varphi}
\end{bmatrix}_{t+1} = 
\begin{bmatrix}
		1 & 0 & \delta\cos\varphi & -\delta\sin\varphi & -\frac{1}{2}\delta^2\cos\varphi & -\frac{1}{2}\delta^2\sin\varphi & 0 & 0
\\		0 & 1 & \delta\sin\varphi & \delta\cos\varphi & \frac{1}{2}\delta^2\sin\varphi & \frac{1}{2}\delta^2\cos\varphi & 0 & 0
\\ 		0 & 0 & 1 & 0 & \delta & 0 & 0 & 0
\\ 		0 & 0 & 0 & 1 & 0 & \delta & 0 & 0
\\ 		0 & 0 & 0 & 0 & 1 & 0 & 0 & 0
\\ 		0 & 0 & 0 & 0 & 0 & 1 & 0 & 0
\\ 		0 & 0 & 0 & 0 & 0 & 0 & 1 & \delta
\\ 		0 & 0 & 0 & 0 & 0 & 0 & 0 & 1
\end{bmatrix}
\begin{bmatrix}
x\\y\\\dot{x}\\\dot{y}\\\ddot{x}\\\ddot{y}\\\varphi\\\dot{\varphi}
\end{bmatrix}_{t}.
}
Where $\delta$ is the total time since the last measurement. See \path{robot_localization/src/predict.cpp} for complete function. Note that (\ref{eqn:state_transition}) is a non-linear system since the state transition matrix depends on $\varphi$.
\newline 

\noindent The estimation error covariance matrix is updated according to:
\eq{
	\tilde P_{t+1} = JPJ^T + \delta Q
}
where $J$ is the Jacobian of the motion model and $Q$ is the process noise covariance (which must be hand tuned).

\subsubsection*{Correction step}
The complete state estimate is corrected using the measurements according to:
\eq{
	s_{t+1} = \tilde{s}_{t+1} + K(m - H\tilde{s}_{t+1})
}
were $m$ is the measurement vector, $H$ is the measurement matrix and $K$ is the Kalman gain given by:
\eq{
	K = \tilde P_{t+1}H^T(H\tilde P_{t+1}+H^T+R)^{-1}.
}
The estimation error covariance matrix is updated according to:
\eq{
	P_{t+1} = (I - KH)\tilde P_{t+1}(I-KH)^T + KRK^T.
}


\end{document}